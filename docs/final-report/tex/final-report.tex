\documentclass[twocolumn]{article}

% Load basic packages
\usepackage{balance}  % to better equalize the last page
\usepackage{graphics} % for EPS, load graphicx instead 
\usepackage{txfonts}
\usepackage{times}    % comment if you want LaTeX's default font
\usepackage{color}
\usepackage{textcomp}
\usepackage{booktabs}
%\usepackage{ccicons}
\usepackage{todonotes}
\usepackage{float}
\usepackage{url}  
\usepackage{titling}	% allows you to move title up the page
\usepackage[pdftex]{hyperref}

% font  sizes
\usepackage{sectsty}			% set font sizes			
\sectionfont{\Large}			% (assumes default font size 10pt)
\subsectionfont{\large}
\subsubsectionfont{\large}
\paragraphfont{\normalsize}

% positioning
\setlength{\parindent}{0em}		% remove indent for new paragraph
\setlength{\parskip}{1em}		% space above paragraph
\setlength{\columnsep}{2em}		% distance between columns
\setlength{\droptitle}{-10em}

% llt: Define a global style for URLs, rather that the default one
% \makeatletter
% \def\url@leostyle{% % \@ifundefined{selectfont}{\def\UrlFont{\sf}}}{\def\UrlFont{\small\bf\ttfamily}}} % \makeatother
% \urlstyle{leo}


\title{Honours Project Report\\Fitness Logger with Micro-services}
\author{Dumitru Vulpe\\BSc (Hons) Applied Computing}
\date{May 2020}

\begin{document}

\maketitle

\begin{abstract}

  This project is a tool to let people be able to log and track workouts overtime easily from their phone in a flexible yet complete way. The initial purpose was to make a phone application which can be used across different disciplines of sport and with different workout types. This was done by breaking up the data in different manageable units so that the use can create their own workflow for logging workouts.

\end{abstract}

\section{Introduction}
\vspace{-1ex}

% use \ref to link to an image.
% this links to \label in the {figure} definition
% I use \ref{fig:xxx} or \ref{tbl:xxx}
% Link to an image at Figure \ref{fig:computerImg}

% IMAGE OF COMPUTER
% to set the location,  \begin{figure}[H]
% \begin{figure}%[]
% \centering
% \includegraphics[width=0.9\columnwidth]{Figures/computer}
% \caption{Computer demo image}~\label{fig:computerImg}
% \end{figure}

\section{Background}
\vspace{-1ex}


\section{Specification}
\vspace{-1ex}


\section{Design}
\vspace{-1ex}

\section{Implementation and Testing}
\vspace{-1ex}

\section{Evaluation}
\vspace{-1ex}

\subsection{Usability}

\subsection{Other Criteria}

\section{Description of the Final Product}
\vspace{-1ex}

\section{Conclusion}
\vspace{-1ex}

\section*{Acknowledgements}
\vspace{-1ex}


\begin{thebibliography}{9}
    % \bibitem{BBC}
    % BBC. Graphs and Charts: Reading Bar Charts. (2011).\\
    % \url{http://www.bbc.co.uk/skillswise/factsheet/ma37grap-e3-f-reading-bar-charts}
\end{thebibliography}

\section*{Appendices}
\vspace{-1ex}

\end{document}
